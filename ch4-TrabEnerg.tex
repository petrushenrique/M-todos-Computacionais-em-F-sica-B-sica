\chapter{Trabalho e Energia}

This question concerns the properties of the potential function found in the file Potential.txt.

a) Plot the potential given by the data

b) Find the equilibrium point(s) of the potential and show if they are stable or unstable

Note

You can use scipy.interpolate.interp1d to create a function which you can then solve using fsolve. Use the xtol parameter to specify a sensible value for the tolerance of the solution found by fsolve.

c) Find and plot the maximum energy of particle bound in this potential

d) Find and plot the allowed region for this bound particle

\begin{lstlisting}[language=Python, frame=lines,basicstyle=\footnotesize, caption={Plot dos dados disponíveis no arquivo Potential.txt}, label={lst:Potential}]
import numpy as np
from matplotlib import pyplot as plt

x, y = np.loadtxt("Potential.txt", unpack=True)

plt.grid()
plt.plot(x, y)
plt.xlabel("x")
plt.ylabel("V(x)")
plt.show()
\end{lstlisting}

\begin{lstlisting}[language=Python, frame=lines,basicstyle=\footnotesize, caption={Análise dos derivadas do potencial dados disponíveis no arquivo Potential.txt}, label={lst:Potential2}]
# Import scipy interpolation
from scipy.interpolate import interp1d
from scipy.optimize import fsolve

# Find first and second derivatives of the potential
derivative = np.gradient(y, x)
second_derivative = np.gradient(derivative, x)

# Create a function using scipy.interpolate.interp1d
f_derivative = interp1d(x, derivative)  # interp1d returns a function given a set of x and y points

# Use fsolve to find zero points
zero_point1 = fsolve(f_derivative, 1)
zero_point2 = fsolve(f_derivative, 2.7, xtol=1e-5)

# Interpolate to find the value of the potential and the second derivative at the turning points
f_potential = interp1d(x, y)
f_second = interp1d(x, second_derivative)

zeros = [zero_point1, zero_point2]

for i, zero in enumerate(zeros):
    # Find values of potential and second derivative
    zero_y = f_potential(zero)
    zero_point_second = f_second(zero)
    
    # Print and plot zero points to 3 s.f.
    print("Zero point %i is as x = %.3f" % (i + 1, zero))
    print("The y value at this point is %.3f" % zero_y)
          
    # Work out if equilibrium is stable
    if zero_point_second < 0:
          print("This is an unstable equilibrium point")
    elif zero_point_second > 0:
          print("This is a stable equilibrium point")
    else:
          print("This equilibrium point is indeterminate. Examine the graph or higher derivatives")
    
    # Plot zero points
    plt.plot(zero, zero_y, "o", color="b")

plt.plot(x, y, label="Potential")
plt.plot(x, derivative, label="1st Derivative")
plt.plot(x, second_derivative, label="2nd Derivative")
plt.grid()
plt.xlabel("x")
plt.legend(loc='center left', bbox_to_anchor=(1, 0.5))  # Put legend outside plot
plt.show()
\end{lstlisting}

Find the maximum energy of a bound particle
The maximum energy of the bound particle is just less than value of the unstable equilibrium point



\begin{lstlisting}[language=Python, frame=lines,basicstyle=\footnotesize, caption={Energia Máxima de uma partícula presa pelo potencial. Dados disponíveis no arquivo Potential.txt},label={lst:Potential3}]
E_max = f_potential(zero_point2)
print("The maximum allowed energy of the particle is %.3f" % E_max)

plt.plot(x, y, label="U(x)")
plt.axhline(E_max, linestyle="--", label="Maximum bound energy", color="k")
plt.grid()
plt.xlabel("x")
plt.legend()
plt.show()
\end{lstlisting}

\begin{lstlisting}[language=Python, frame=lines,basicstyle=\footnotesize, caption={Região permitida para a partícula presa pelo potencial. Dados disponíveis no arquivo Potential.txt},label={lst:Potential4}]
# Solve for the point where the lines U(x) and E_max intersect
# i.e. U(x) - E_max = 0

def f_intersection(x_in):
    intersect = f_potential(x_in) - E_max
    return intersect

x1 = fsolve(f_intersection, 0.5, xtol=1e-5)
x2 = zero_point2

print("The particle is bound between the points %.3f and %.3f" % (x1, x2))

plt.plot(x, y, label="U(x)")
plt.axhline(E_max, linestyle="--", label="Maximum bound energy", color="k")
plt.axvline(x1, linestyle="--", color="k")
plt.axvline(x2, linestyle="--", color="k")
plt.plot([x1, x2], [E_max, E_max], "o", color="b")
plt.grid()
plt.xlabel("x")
plt.legend()
plt.show()
\end{lstlisting}

Efeito resistivo do ar em bicicletas
lançamento de projéteis
ver \\  \href{URL}{https://drive.google.com/file/d/13ts72fG1Y2DRGR2Zralu3o-S1tEZ88sp/view}

