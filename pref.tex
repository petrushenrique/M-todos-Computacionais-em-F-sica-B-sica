	\chapter{Prefácio}
\epigraph{If you want to master something, teach it.}{Richard Feynman}
	
	A idéia central deste material didático é servir como um complemento as disciplinas de Física Básica (normalmente chamadas de Física I a Física IV) abordando como utilizar computação em Física. É portanto uma disciplina diferente das disciplinas ``Física Computacional'' que tem aparecido nos currículos das graduações em Física pelo Brasil. A ideia é que desde o início da sua graduação seja natural para o futuro Físico empregue computação nas suas atividades. Na UFCAT, isso é feito via 4 disciplinas que acontecem em paralelo com as disciplinas de Física Básica. Em princípio, os alunos já cursaram as disciplina ``Algorítimos e Programação de Computadores''.
	
	Os capítulos apresentam problemas de Física em ordem semelhante a livros tradicionais como \cite{halliday2002fundamentals} e \cite{nussenzveig2013curso, nussenzveig2014curso, nussenzveig2015curso, nussenzveig2018curso}. Tópicos a respeito de métodos computacionais foram colocados nos apêndices e devem muito a referências que já utilizam Python como \cite{ayars2013computational, hill2020learning}. Também haverá um apêndice tratando do uso de \LaTeX.
	
	Os cursos geralmente poderão começar pelo capítulo 1 e utilizar alguns dos apêndices. No demais as ementas da UFCAT sugerem a seguinte disposição:
	\begin{enumerate}
	    \item  {\bf Métodos Computacionais em Física 1:}  Capítulos 2, 3, 4, 5, 6. 
	    No fluxo normal a disciplina de Física 1 é feita no segundo período. Os alunos estarão cursando as disciplinas ``Física 1'' e ``Cálculo Numérico'' em paralelo a esta disciplina.
	    \item {\bf Métodos Computacionais em Física 2:} Capítulos 7, 8, 9, 10, 11. No fluxo normal a disciplina de Física 1 é feita no terceiro período. Os alunos estarão cursando em paralelo as disciplinas ``Física 2'', ``Probabilidade e Estatística'' e ``Cálculo Vetorial''.
	    \item {\bf Métodos Computacionais em Física 3:} Capítulos 12, 13, 14. No fluxo normal a disciplina de Física 1 é feita no quarto período. Os alunos estarão cursando em paralelo as disciplinas ``Física 3'', ``Cálculo III'' (equações diferenciais ordinárias) e ``Física Matemática 1'' (o que permite uso mais extensivo de números complexos).
	    \item {\bf Métodos Computacionais em Física 4:} Capítulos 15, 16, 17. No fluxo normal a disciplina de Física 1 é feita no quinto período. Os alunos estarão cursando em paralelo a disciplina ``Física 4''.
	\end{enumerate}
	
O material esta sendo escrito em paralelo ao período em que ministro as disciplinas e portanto esta sendo escrito de maneira não sequencial, obedecendo a demanda de oferta do IF/UFCAT. A tabela abaixo mostra a estimativa para conclusão de cada capítulo:

\begin{table}[h]
    \centering
    \begin{tabular}{cll}
        Capítulo & Título & ETA  \\
        1 & Introdução ao Python & 09/05/2022 \\
        2 & Cinemática & 10/10/2022 \\
        3 & Dinâmica & 17/10/2022 \\ 
        4 & Trabalho e Energia & 24/10/2022 \\
        5 & Momento Linear & 31/10/2022 \\
        6 & Rotações, Torque e Momento Angular & 17/10/2022 \\
        7 & Gravitação & 16/05/2022 \\
        8 & Oscilador Harmônico & 23/05/2022 \\
        9 & Ondas & 30/05/2022 \\
        10 & Fluidos & 06/06/2022 \\
        11 & Termodinâmica e Teroia Cinética dos Gases & 13/06/2022\\
        12 & Campo Elétrico & 10/10/2022 \\
        14 & Campo Magnético & 24/10/2022 \\
        15 & Equações de Maxwell e Oscilações Eletromagnética & 07/11/2022\\
        16 & Óptica Física e Geométrica & 21/11/2022 \\
        17 & Princípio de Fermat e Outros Problemas de Extremização & 05/12/2022 \\
        A  & Elementos de Python & 20/06/2022 \\
        B  & Phyton em Aplicações Numéricas: Numpy & 27/06/2022 \\
        C  & Construção de Gráficos com Matplotlib & 04/07/2022 \\
        D  & Computação Algébrica com Sympy & 11/07/2022 \\
        E  & Elementos de Cálculo Numérico & 18/07/2022 \\
        F  & Usando \LaTeX & 25/07/2022
    \end{tabular}
    \caption{Cronograma previsto para a produção dos Capítulos.}
\end{table}