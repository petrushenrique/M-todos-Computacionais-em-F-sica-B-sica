\chapter{Momento Linear}
Problemas do Haliday Capítulos 9

\begin{lstlisting}[language=Python, frame=lines,basicstyle=\footnotesize, caption={Determinando Centro de Massa de uma distribuição aleatória em 2D}, label={lst:2DCM}]
import numpy as np
import matplotlib.pyplot as plt
from mpl_toolkits.mplot3d import Axes3D

n =100
# Gerar posicoes e Massas aleatorias
x = np.random.randint(-50, 50, n)
y = np.random.randint(0,200,n)
m = np.random.randint(1,200, n)

# Calcula as Coordenadas do Centro de Massa
cgx = np.sum(x*m)/np.sum(m)
cgy = np.sum(y*m)/np.sum(m)

# Plot do Centro de Massa
plt.scatter(x,y,s=m);
plt.scatter(cgx, cgy, color='k', marker='+', s=1e4);
plt.title('Centro de Massa');
\end{lstlisting}

