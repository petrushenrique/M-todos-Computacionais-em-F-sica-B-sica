\chapter{Termodinâmica e Teoria Cinética dos Gases}
Problemas do Haliday Capítulo 18 a 20

1. A soap bubble with surface tension $\gamma = 2.50 \times 10^2 N/m$
has a radius $r_0 = 2.0 mm$ when the pressure outside the bub-
ble is 1.0 atmosphere. (a) Numerically calculate the radius of
the soap bubble when the pressure outside the bubble drops to
0.5 atm. (b) Numerically calculate the radius of the soap bub-
ble if the pressure outside the bubble is raised to 2.0 atm.
2. A small balloon is filled with nitrogen gas (assumed ideal) at
the bottom of the Marianas Trench, 35,000 ft beneath the sur-
face of the ocean. The balloon originally has a radius of 1.0
mm, is massless, and is infinitely expandable without any sur-
face tension, but always keeps a spherical shape. Assume the
ideal gas inside the balloon is at $4^o$ C throughout this problem.
The balloon begins to rise to the surface, as the balloon rises
it expands, and as it moves there is a retarding force f propor-
tional to speed v and balloon radius r given by
\[f = 6\pi \eta rv,\]
where $\eta = 1.7 \times 10^{-3} N s/m$ is the viscosity of water. (a)
Calculate the initial buoyant force on the balloon. (b) What
will be the size of the balloon on the surface? (c) Numerically
solve this problem to find out how long it takes for the bal-
loon to rise to the surface.

1. Write a program to simulate the random walk of a particle.
The particle starts at the origin, and can then take a step with
$\Delta x$ and $\Delta y$ increments assigned randomly between $-1$ and $+1$.
(a) Allow the particle to “walk” through 200 steps, and graph
the motion. Choose the scale of the
graph to just fit the data. (b) Allow the particle to walk
through 2000 steps, but this time plot the position of the particle only at the end of each 10 steps. Again, choose the scale
of the graph to just fit the data. (c) Repeat, but now allow the
particle to walk through 20,000 steps, and only plot the position at the end of each 100 steps. Compare the three graphs.
How does the size of the graph grow with the number of
steps? Do the graphs look similar? If the graphs were shuffled, would you be able to tell which was which?

2. Consider a van der Waals gas with $a = 0.10 J m^3 /mol$ and
$b = 1.0 \times 10^4 m^3 /mol$. (a) Find the temperature $T_{cr}$ , pressure $p_cr$ , and volume $V_cr$ where $p/V 0$ and $2 p/V20$.
(b) Graph the pressure along isotherms as a function of vol-
ume for 0.80T cr , 0.85T cr , 0.90T cr , 0.95T cr , 1.00T cr , 1.05T cr ,
and 1.10T cr . The graphs should extend from $V 0$ to $V$ 
5V cr . (c) What is physically significant about the T cr isotherm?

Difusão da Molécula de um gás

\begin{lstlisting}[language=Python, frame=lines,basicstyle=\footnotesize, caption={Difusão da Molécula de Um Gás em 2D}, label={lst:Diffusion}]
import numpy as np
import matplotlib.pyplot as plt
import pylab as pl
from itertools import cycle


pi = np.pi
# Simular os passos
def random_walk(step_n, step_size):
    origin = np.zeros((1)) # comece em 0
    steps = 2*pi*np.random.uniform(low=0.0, high=1.0, size=step_n)
    x_steps = step_size*np.cos(steps)
    y_steps = step_size*np.sin(steps)
    x_path = np.concatenate([origin, x_steps]).cumsum(0)
    y_path = np.concatenate([origin, y_steps]).cumsum(0)
    return x_path, y_path


step_n = 100000
step_size = 1

x_path, y_path = random_walk(step_n, step_size)
xstart = x_path[:1]
ystart = y_path[:1]
xstop = x_path[-1:]
ystop = y_path[-1:]

fig = pl.figure(figsize=(8,4),dpi=200)
ax = fig.add_subplot(111)
ax.plot(x_path, y_path,c='blue',alpha=0.5,lw=0.5,ls='-');
ax.plot(xstart, ystart, c='red', marker='+')  #Plot o inicio
ax.plot(xstop, ystop, c='black', marker='o') # Plot o Final
pl.title('2D Random Walk')
pl.tight_layout(pad=0)
pl.grid(which="both")
pl.savefig('random_walk_1d.png',dpi=250);

\end{lstlisting}
