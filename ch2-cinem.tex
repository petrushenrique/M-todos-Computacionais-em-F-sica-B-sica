\chapter{Cinemática}
Problemas do Haliday Capítulos 2 a 4

\begin{lstlisting}[language=Python, frame=lines,basicstyle=\footnotesize, caption={Lançamento de Projeteis}, label={lst:projeteis1}]
from math import *
import matplotlib.pyplot as plt

#Input
theta = float(input('Qual o angulo de Lancamento? (em graus)'))
v = float(input('Qual a velocidade de Lancamento? (em m/s )'))
theta = pi*theta/180

# Condicao Inicial
x0 = 0
y0 = 0
vx0 = v*cos(theta)
vy0 = v*sin(theta)

step = 200
g = -9.8
t = [i*0.01 for i in range(step+1)]
x = [i*vx0 + x0 for i in t]
y = [ g/2*i**2 + vy0*i+y0 for i in t]

plt.scatter(x,y)
plt.xlabel('x')
plt.ylabel('y')
plt.show()
\end{lstlisting}

\section{Movimento Circular}


\begin{lstlisting}[language=Python, frame=lines,basicstyle=\footnotesize, caption={Movimento da Terra e Lua ao Redor do Sol}, label={lst:earth-moon1}]

import numpy as np
import matplotlib.pyplot as plt
plt.rcParams["figure.figsize"] = (8,8)

pi = np.pi


#Planetary data
earth_sn = 149597870
moon_earth =384400
R_e = earth_sn/moon_earth
R_m = 1
T_e = 12 # Lunar periods
T_m = 1 # Lunar periods
step = 0.01
print(R_m/R_e)

t = np.linspace(0,T_e, int(T_e/0.01))

#Earth motion

x_e = R_e*np.cos(2*pi*t/T_e)
y_e = R_e*np.sin(2*pi*t/T_e)

# moon motion

x_m = x_e+R_m*np.cos(2*pi*t/T_m)
y_m = y_e+R_m*np.sin(2*pi*t/T_m)


plt.plot(x_e[20:90], y_e[20:90])
plt.plot(x_m[20:90], y_m[20:90])

\end{lstlisting}



{\bf Problema 1:}
Uma bola é lançada de uma torre de altura $h$ com velocidade inicial zero. Escreva um
programa que pede ao usuário para digitar a altura em metros da torre e depois calcula
e imprime o tempo que a bola leva até atingir o solo, ignorando a resistência do ar. Use
seu programa para calcular o tempo para uma bola cair de uma torre de 100 m de altura

