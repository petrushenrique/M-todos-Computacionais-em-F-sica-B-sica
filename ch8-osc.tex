\chapter{O Oscilador Harmônico}
Problemas do Haliday Capítulo 15

\begin{lstlisting}[language=Python, frame=lines, basicstyle=\footnotesize, caption={Oscilador Harmônico Simples}, label={lst:OHS}]
import matplotlib.pyplot as plt
import numpy as np

# set constants
k = 10
m = 1
t_max = 10.0

#Set the number of iterations to be used in the for loop
no_of_iterations=100000

# set time step so that the loop will always iterate until t=t_max seconds 
dt = t_max/no_of_iterations


# make arrays to store data
t = np.zeros(no_of_iterations)
x = np.zeros(no_of_iterations)
v = np.zeros(no_of_iterations)

# set initial conditions
t[0] = 0
x[0] = 5
v[0] = 0


# evolve
for i in np.arange(1,no_of_iterations):
    t[i] = dt * i
    v[i] = v[i-1] - dt *k/m*x[i-1]
    x[i] = x[i-1] + dt * v[i]

fig, ax = plt.subplots(2, 2,figsize=(10, 10))
ax[0,0].plot(t, x)
ax[0,0].set(title = 'time x position', xlabel='time', ylabel='position')
ax[1,0].plot(t, v)
ax[0,0].set(title = 'time x velocity', xlabel='time', ylabel='velocity')
ax[0,1].plot(x,v)
ax[0,1].set(title = 'position x velocity', xlabel='position', ylabel='velocity')
ax[1,1].plot(t,1/2*m*v**2+1/2*k*x**2)
ax[1,1].set(title = 'time x energy', xlabel='time', ylabel='energy')

\end{lstlisting}

Consider a mass of 1 Kg on a spring of stiffness of $1 Nm^{-1}$ that is displaced by 1 cm and then released. Calculate and then plot on the same graph the kinetic, potential and total energies over a period of several oscillations. Now consider the same spring, but with a damping force of 0.4 N opposing the oscillations. Plot the kinetic, potential and total energies of this system.



\begin{lstlisting}[language=Python, frame=lines, basicstyle=\footnotesize, caption={Oscilador Harmônico Duplo}, label={lst:OHS2}]
import matplotlib.pyplot as plt
import numpy as np

# set constants
k = 10
m = 1
t_max = 10.0
l = 1
#Set the number of iterations to be used in the for loop
no_of_iterations=1000

# set time step so that the loop will always iterate until t=t_max seconds 
dt = t_max/no_of_iterations


# make arrays to store data
t = np.zeros(no_of_iterations)
x1 = np.zeros(no_of_iterations)
v1 = np.zeros(no_of_iterations)
x2 = np.zeros(no_of_iterations)
v2 = np.zeros(no_of_iterations)

# set initial conditions
t[0] = 0
x1[0] = 0
v1[0] = 5
x2[0] = 0
v2[0] = -5


# evolve
for i in np.arange(1,no_of_iterations):
    t[i] = dt * i
    v1[i] = v1[i-1] - dt *k/m*(x1[i-1]-x2[i-1])
    v2[i] = v2[i-1] + dt *k/m*(x1[i-1]-x2[i-1])
    x1[i] = x1[i-1] + dt * v1[i]
    x2[i] = x2[i-1] + dt * v2[i]

fig, ax = plt.subplots(2, 2,figsize=(10, 10))
ax[0,0].plot(t, x1,label="particle 1")
ax[0,0].plot(t, x2,label="particle 2")
ax[0,0].set(title = 'time x position', xlabel='time', ylabel='position')
ax[0,0].legend(loc='upper left')
ax[1,0].plot(t, v1,label="particle 1")
ax[1,0].plot(t, v2,label="particle 2")
ax[1,0].set(title = 'time x velocity', xlabel='time', ylabel='velocity')
ax[1,0].legend(loc='upper left')
ax[0,1].plot(t,(1/2)*m*v1**2+(k/2)*x1**2)
ax[0,1].plot(t,(1/2)*m*v2**2+(k/2)*x2**2)
ax[0,1].set(title = 'time vs energy', xlabel='time', ylabel='energy')
ax[1,1].plot(t,(x1+x2)/2)
ax[1,1].set(title = 'Movement of the center of mass', xlabel='time',
            ylabel='Center of Mass')
\end{lstlisting}

\begin{lstlisting}[language=Python, frame=lines, basicstyle=\footnotesize, caption={Rotores acoplados}, label={lst:OH-Rotores}]
import matplotlib.pyplot as plt
import numpy as np

# parameters

damping = 0.1
freq_nl = 5.0
total_time = 20
dt = 0.0001
steps = int(total_time/dt)
pi = np.pi

#arrays

theta1 = np.zeros(steps)
theta2 = np.zeros(steps)
v1 =  np.zeros(steps)
v2 =  np.zeros(steps)
t = dt*np.arange(steps)

# Initial conditions
theta1[0] = 0
theta2[0] = pi/3
v1[0] = 0
v2[0] = 0


for i in range(steps-1):
    theta1[i+1] = theta1[i] + dt*v1[i]
    theta2[i+1] = theta2[i] + dt*v2[i]
    v1[i+1] = v1[i] + dt*(freq_nl*np.sin(theta1[i+1]-theta2[i+1]) - damping*v1[i+1])
    v2[i+1] = v2[i] + dt*(freq_nl*np.sin(theta2[i+1]-theta1[i+1]) - damping*v2[i+1])

# Plots
plt.plot(t,theta1, label = "rotor 1")
plt.plot(t, theta2, label = "rotor 2")
plt.legend()
plt.show()
\end{lstlisting}

Consider two pendula of mass 1 Kg attached to each other by a spring of stiffness $1 Nm^{-1}$ and suspended from ideal strings of length 1 m. The left mass is displaced from equilibrium by 1 cm whilst the right one is held fixed, and then released. Use the fourth order Runge-Kutta method (explained in the solutions to problem 1) to simulate the displacement from equilibrium of each mass as the system evolves over time. Now extend the method so that it will solve for N identical pendula attached by identical springs in a line, with the same displacement of the left pendulum to start it off.

1. Consider a system composed of two objects constrained
to move along the x axis. The first object is connected to
a spring that is attached to the origin, and the second object
is connected to a spring that is attached to the first object.
Both objects have the same mass, 0.10 kg, and both springs
have the same force constant, 1.0 N/m. (a) Numerically
simulate the motion of the objects, assuming that the second
object is pulled and then released a distance of 1.0 cm
from the equilibrium position. Generate a graph of the mo-
tion of the objects. (b) Use a fast Fourier transform (avail-
able on some spread sheet programs) to show that there
are two characteristic frequencies of the motion. What are
these frequencies?

2. An object of mass m moves subject to a force that results in a
potential energy of U(x) = 14 kx 4 . This type of motion is called
a quartic oscillator. Note that the frequency of oscillation de-
pends on the amplitude of the oscillations here. Assuming a
mass m = 0.10 kg and a force constant of k = 100 N/m 3 , nu-
merically simulate the motion for several different ampli-
tudes. Graph the results, and find the relation between ampli-
tude and frequency for this system.

Pêndulo Simples além das pequenas oscilações
ver  \href{URL}{https://drive.google.com/file/d/13ts72fG1Y2DRGR2Zralu3o-S1tEZ88sp/view}

