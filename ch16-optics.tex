\chapter{Óptica Física e Geométrica}

If all is in phase the wavefront is parallel to $x$. If we now want an angle $\theta$ we have $d=\eta \sin \theta$. The wave propagates with $\cos (\omega t - k r)$, and we may set $t=0$. So if we want the point at the end of $d$ being in phase the source at $\eta$ must be at phase $\phi = kd$. The left point in the sketch I set as phase $\phi = 0$. I am free to do so and I decide to measure all with respect to this point. This point, hence, also defines the wave front. Now the point at $\eta$ gives me a wave $A\cos(kr)$, were $r=\sqrt{(x - \eta)^2+y^2}$, but wait: I want it to have a phase so it is $A \cos(kr -\phi(x))$. Now I want this wave to have a phase zero at the wave front (sort of by definition). In other words the argument of the cosine must be zero for $r=d$, i.e. $kd(\eta) - \phi(\eta)=0$. Only then this is part of the wavefront. So $kd(\eta)=\phi(\eta)$ and plugging $d=\eta \sin \theta$ gives, hence, $\phi=k\eta \sin \theta$.

\begin{lstlisting}[language=Python, frame=lines,basicstyle=\footnotesize, caption={Princípio de Huygens}, label={lst:Huygens}]
import matplotlib
import matplotlib.pyplot as plt
import numpy as np

def f(x,y,phi=0): return np.cos(2*np.pi*np.sqrt(x**2+y**2)-phi)

def wavephase(x,y,theta=0,pnts=100):
    out=0
    for pt in np.linspace(-20,20,pnts):
        out+=f(x-pt,y,2*np.pi*pt*np.sin(theta))
    return out

degree=np.pi/180.

n = 120
x = np.linspace(-10,10,2*n)
y = np.linspace(-10,0,n)
X,Y = np.meshgrid(x,y)

fff=15
fig=plt.figure()
ax=fig.add_subplot(3,1,1)
bx=fig.add_subplot(3,1,2)
cx=fig.add_subplot(3,1,3)
ax.contourf(X, Y, f(X,Y), 22, alpha=1, cmap="Blues")
bx.contourf(X, Y, wavephase(X,Y, theta=0*degree), 22, alpha=1, cmap="Blues")
bx.scatter(np.linspace(-20,20,100)[25:-25],np.linspace(-20,20,100)[25:-25]*0,
    c='b',s=45)
cx.contourf(X, Y, wavephase(X,Y, theta=fff*degree), 22, alpha=1, cmap="Blues")
cx.plot([0,10*np.cos(fff*degree)],[0,-10*np.sin(fff*degree)],color='#ffaa00')
cx.scatter(np.linspace(-20,20,100)[25:-25],np.linspace(-20,20,100)[25:-25]*0,
    c=np.linspace(-20,20,100)[25:-25],s=45)
for ma in [ax,bx,cx]:
    ma.set_xlim([-10,10])
    ma.set_ylim([-10,0])
plt.show()
\end{lstlisting}

Problemas do Haliday Capítulos 34 a 36
